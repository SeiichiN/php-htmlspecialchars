\documentclass[uplatex, dvipdfmx]{jsarticle}


\usepackage{tcolorbox}
\usepackage{color}
\usepackage{listings, plistings}

%% ノート/latexメモ
%% http://pepper.is.sci.toho-u.ac.jp/pepper/index.php?%A5%CE%A1%BC%A5%C8%2Flatex%A5%E1%A5%E2

% Java
\lstset{% 
  frame=single,
  backgroundcolor={\color[gray]{.9}},
  stringstyle={\ttfamily \color[rgb]{0,0,1}},
  commentstyle={\itshape \color[cmyk]{1,0,1,0}},
  identifierstyle={\ttfamily}, 
  keywordstyle={\ttfamily \color[cmyk]{0,1,0,0}},
  basicstyle={\ttfamily},
  breaklines=true,
  xleftmargin=0zw,
  xrightmargin=0zw,
  framerule=.2pt,
  columns=[l]{fullflexible},
  numbers=left,
  stepnumber=1,
  numberstyle={\scriptsize},
  numbersep=1em,
  language={Java},
  lineskip=-0.5zw,
  morecomment={[s][{\color[cmyk]{1,0,0,0}}]{/**}{*/}},
  keepspaces=true,         % 空白の連続をそのままで
  showstringspaces=false,  % 空白字をOFF
}
%\usepackage[dvipdfmx]{graphicx}
\usepackage{url}
\usepackage[dvipdfmx]{hyperref}
\usepackage{amsmath, amssymb}
\usepackage{itembkbx}
\usepackage{eclbkbox}	% required for `\breakbox' (yatex added)
\usepackage{enumerate}
\usepackage[default]{cantarell}
\usepackage[T1]{fontenc}
\usepackage{endnotes}
\fboxrule=0.5pt
\parindent=1em

\makeatletter
\def\verbatim@font{\normalfont
\let\do\do@noligs
\verbatim@nolig@list}
\makeatother

\begin{document}

%\anaumeと入力すると穴埋め解答欄が作れるようにしてる。\anaumesmallで小さめの穴埋めになる。
\newcounter{mycounter} % カウンターを作る
\setcounter{mycounter}{0} % カウンターを初期化
\newcommand{\anaume}[1][]{\refstepcounter{mycounter}{#1}{\boxed{\phantom{aa}\textnormal{\themycounter}\phantom{aa}}}} %穴埋め問題の空欄作ってる。
\newcommand{\anaumesmall}[1][]{\refstepcounter{mycounter}{#1}{\boxed{\tiny{\phantom{a}\themycounter \phantom{a}}}}}%小さい版作ってる。色々改造できる。

%% 修正時刻: Wed Feb 16 06:54:13 2022



\section{htmlspecialchars関数の働き}

htmlspecialchars関数の働きは、以下のようなものである。

\begin{lstlisting}[numbers=none, language=php]
 $htmltext = '<div id="wrap"><h1>TEST</h1></div>';
 echo htmlspecialchars($htmltext, ENT_QUOTES, "UTF-8");
\end{lstlisting}

echo関数で出力された文字列は、以下である。

\vspace{3mm}
\begin{tabular}{|c|} \hline
\verb!&lt;div id=&quot;wrap&quot;&gt;&lt;h1&gt;TEST&lt;/h1&gt;&lt;/div&gt;! \\ \hline
\end{tabular}
\vspace{3mm}

これをブラウザで見ると、 \\
\fbox{<div id="wrap"><h1>TEST</h1></div>} \\
 となっている。

 だから、フォームにて、JavaScript や $<$table$>$タグなどの余計な HTMLタグが
 入力されたとしても、それを無力化できる。



 \section{PHP7+MySQL入門ノートでの記述}

『PHP7+MySQL入門ノート』では、p266 で、「htmlspesialchars()を便利に使うためのユーザ定義関数 es()」として以下のような説明がある。

\begin{tcolorbox}
 ユーザからのデータをブラウザに表示する前に htmlspecialchars() を通して
 HTMLエスケープを行うことが必須となりますが、この処理を行うために
 array\_map() をうまく利用したユーザ定義関数を作っておくと便利です。
 
 ...( 略 )...

 こうすることで、es() は引数が1個の値でも配列でも htmlspecialchars() で
 処理できる関数になります。
\end{tcolorbox}

この説明のあとに、以下のコードが紹介されている。
 
\begin{lstlisting}[caption=util.php, language=php]
<?php
// XSS対策のためのHTMLエスケープ
function es($data, $charset='UTF-8'){
  // $dataが配列のとき
  if (is_array($data)){
    // 再帰呼び出し
    return array_map(__METHOD__, $data);
  } else {
    // HTMLエスケープを行う
    return htmlspecialchars($data, ENT_QUOTES, $charset);
  }
}
\end{lstlisting}

その後に es()関数をテストするコードが紹介されている。
このコードを実行することで、htmlspecialchars関数の働きを確認することが
できる。

で、この es()関数を実際に使用したコードは、p272の nameCheck.php である。

\begin{lstlisting}[caption=nameCheck.php, language=php]
 <!DOCTYPE html>
 <html lang="ja">
 <head>
   <meta charset="UTF-8">
   <title>フォーム入力チェック</title>
   <link href="../../css/style.css" rel="stylesheet">
 </head>
 <body>
   <div>

   <?php
     reuqire_once("../../lib/util.php");
     // 文字エンコードの検証
     if (!cken(%_POST)) {
       $encoding = mb_internal_encoding();
       $err = "Encoding Error! The expected encoding is " . $encoding ;
       // エラーメッセージを出して、以下のコードをすべてキャンセルする
       exit($err);
     }
     // HTMLエスケープ(XSS対策)
     $_POST = es($_POST);                             // <== 
   ?>

   ... ( 以下、略 ) ...
     
\end{lstlisting}

ここでは、\$\_POST 連想配列の中の文字列を HTMLエスケープした後、
すぐに画面に出力している。
 
この著者のやり方では、\$\_POST データが送られてきたら、まず、「文字エンコードの検証」を
おこない(14行目)、次に「HTMLエスケープ」をおこなっている(21行目)。

「文字エンコードの検証」は必要だと思うが、

\vspace{3mm}
{\Large ``HTMLエスケープ''は \$\_POSTデータを取得直後にするべきだろうか ?}
\vspace{3mm}

特に問題だと思われるのは、21行目である。

\fbox{\$\_POST = es(\$\_POST)}

\vspace{3mm}
{\Large \$\_POST の中味を書き変えている !}
\vspace{3mm}

この es関数がどのようなものかというと、
\$\_POST の中を再帰的に htmlspecialchars関数を実行するというものである。

たとえば、以下のような \$\_POST データが送られてきたとする。

\begin{lstlisting}[numbers=none, language=php]
 $_POST = [
  'name' => '<textarea>悪意</textarea>',
  'text' => '<script>alert("virus")</script>'
 ];
\end{lstlisting}

\verb!$_POST = es($_POST)!とすることで、
\$\_POSTの中身を、以下のようなHTMLエスケープされた文字列に置き変えている。
\endnote{\$\_POSTの中身}

\begin{lstlisting}[numbers=none, language=php]
Array
(
    [name] => &lt;textarea&gt;悪意&lt;/textarea&gt;
    [text] => &lt;script&gt;alert(&quot;virus&quot;)&lt;/script&gt;
)
\end{lstlisting}

ここでは、\$\_POST の中味をすぐに画面に出力しているからいいが、
これを MySQL などに保存するとなると、大事になる。

\section{このやり方の良くないところ}

ここでの著者のやり方は、\$\_POST でデータが送られてきたら、とりあえず、
htmlspecialchars関数を使って \$\_POST を安全なものにしてしまおうという
ものである。

初心者の人にこのやり方を教えれば、この通りにすぐに htmlspecialchars関数を使って
同じようにやってしまうだろう。

しかし、本来は、htmlspecialchars関数は、画面に出力するタイミングで行うもので
なければならない。
この著者のやり方では、間違ったタイミングを教えてしまうことになる。

更に良くないのは、\$\_POST を書き変えてしまう点である。
元のデータは大事にしなければならない。これは避けるべきだと思う。


\section{この著者は実は、すぐにHTMLエスケープすべきだとは思っていない}

この著者は、本当は、\$\_POSTデータを取得後すぐに HTMLエスケープすべきだと
は思っていないのは、以下のコードを見るとわかる。


p503の''insert\_member.php''である。

\begin{lstlisting}[
 caption=insert\_member.php,
 language=php
]
 <?php
 require_once("../../lib/util.php");
 $gobackURL = "insertform.html";

 // 文字エンコードの検証
 if (!cken($_POST)) {
   header("Location:($gobackURL)");
   exit();
 }                                       // <== <1> es()関数は使っていない

 // 簡単なエラー処理
 $errors = [];
 if (!isset($_POST["name"]) ||($_POST["name"] === "")) {
   $errors[] = "名前が空です。";
 }
 if (!isset($_POST["age"]) ||(!ctype_digit($_POST["age"] === ""))) {
   $errors[] = "年齢には数値を入れてください。";
 }
 if (!isset($_POST["sex"]) || !in_array($_POST["age"], ["男", "女"])) {
   $errors[] = "性別が男または女ではありません。";
 }

// エラーがあったとき
 if (count($errors) > 0) {
   echo '<ol class="error">';
   foreach ($errors as $value) {
     echo "<li>", $value, "</li>";
   }
   echo "</ol>";
   echo "<hr>";
   echo "<a href=", $gobackURL, ">戻る</a>";
   exit();
}
 
 // データベースユーザ
 $user = 'testuser';
 $password = 'pw4testuser';
 // 利用するデータベース
 $dbName = 'testdb';
 // MySQLサーバ
 $host = 'localhost:8889';
 // MySQLの DSN文字列
 $dsn = "mysql:host={$host};dbname={$dbname};charset=utf8";
 ?>

 <!DOCTYPE html>
 <html lang="ja">
 <head>
 <meta charset="utf-8">
 <title>レコード追加</title>
 <link href="../../css/style.css" rel="stylesheet">
 <!-- テーブル用のスタイルシート -->
 <link href="../../css/tablestyle.css" rel="stylesheet">
 </head>
 <body>
 <div>
   <?php
   $name = $_POST["name"];                          //  <== <2> es()関数は使っていない
   $age = $_POST["age"];
   $sex = $_POST["sex"];
   // MySQLデータベースに接続する
   try {
     $pdo = new PDO($dsn, $user, $password);
     // プリペアドステートメントのエミュレーションを無効にする
     $pdo->setAttribute(PDO::ATTR_EMULATE_PREPARES, false);
     // 例外がスローされる設定にする
     $pdo->setAttribute(PDO::ATTR_ERRMODE, PDO::ERROMDE_EXCEPTION);
     // SQL文をつくる
     $sql = "INSERT INTO member (name, age, sex) VALUES (:name, :age, :sex)";
     // プリペアドステートメントを作る
     $stm = $pdo->prepare($sql);
     // プレースホルダに値をバインドする
     $stm->bindValue(':name', $name, PDO:PARAM_STR);           // <== <3> es()関数は
     $stm->bindValue(':age', $age, PDO:PARAM_INT);             //       使っていない
     $stm->bindValue(':sex', $sex, PDO:PARAM_STR);
     // SQL文を実行する
     if ($stm->execute()) {
       // レコード追加後のレコードリストを取得する
       $sql = "SELECT * FROM member";
       // プリペアドステートメントを作る
       $stm = $pdo->prepare($sql);
       // SQL文を実行する
       $stm->execute();
       // 結果の取得(連想配列で受け取る)
       $result = $stm->fetchAll(PDO::FETCH_ASSOC);
       // テーブルのタイトル行
       echo "<table>";
       echo "<thead><tr>";
       echo "<th>", "ID", "</th>";
       echo "<th>", "名前", "</th>";
       echo "<th>", "年齢", "</th>";
       echo "<th>", "性別", "</th>";
       echo "</tr></thead>";
       // 値を取り出して行に表示する
       echo "<tbody>";
       foreach ($result as $row) {
         // 1行ずるテーブルに入れる
         echo "<tr>";
         echo "<td>", es($row['id']), "</td>";                  // <== <4> ここで
         echo "<td>", es($row['name']), "</td>";                //    es()関数を使って
         echo "<td>", es($row['age']), "</td>";                 //    いる。
         echo "<td>", es($row['sex']), "</td>";
         echo "</tr>";
       }
       echo "</tbody>";
       echo "</table>";
     } else {
       echo '<span class="error">追加エラーがありました。</span><br>';
     }
   } catch (Exception $e) {
     echo "<span class="error">エラーがありました。</span><br>";
     echo $e->getMessage();
   }
   ?>
   <hr>
   <p><a href="<?php echo $gobackURL ?>">戻る</a></p>
 </div>
 </body>
 </html>
\end{lstlisting}

\noindent
\verb!<1>! 9行目 \\
このコードでは、``insertform.html''から送られてきた \$\_POST['name']データ、
\$\_POST['age']データ、\$\_POST['sex']データについて、
6行目でエンコードチェックしているが、es()関数は使っていない。\\

\noindent
\verb!<2>! 58〜60行目 \\
\$\_POST['name'] は \$name という変数に、\$\_POST['age'] は \$age という変数に、
\$\_POST['sex'] は \$sex という変数に、格納している。
その際にも、es()関数は使っていない。\\

\noindent
\verb!<3>! 73行目〜75行目 \\
この \$name、\$age、\$sex はSQL文に埋めこまれて、データベースに登録されている。
その際にも、es()関数は使われていない。\\

\noindent
\verb!<4>! 99行目〜102行目 \\
ここではじめて es()関数が使われている。

この部分は、MySQLデータベースから取り出したデータを画面に出力している。
このデータはデータベースに登録されている全てのデータである。

それらのデータをデータベースに格納する際に、es()関数を使って
HTML文字列やJavaScriptを無効化させることはやっていない。
あくまで、画面に出力する間際で es()関数を実行している。\\

\section{なぜこんなコードを初心者に勧めているのか?}

著者は、htmlspecialchars関数の意味も使いどころもわかっているが、
では、なぜ \fbox{\$\_POST = es(\$\_POST)} なんてことを初心者に
すすめているのだろうか?

「初心者には難しいことを言ってもだめだから、とりあえず \$\_POST を
エスケープさせとけ」ということだろうか?

p272 の nameCheck.php など、掲載されているサンプルコードのほとんどは、
\$\_POST からデータを取得して、あとは画面に出力する
だけだから、\fbox{\$\_POST = es(\$\_POST)} としても問題はないわけで
ある。画面出力するところで、いちいち \fbox{echo \ es(\$name)} とやるよりは
楽である。

画面に主力するだけではなしに、データベースに保存したり、他の処理に \$\_POST
を使うなら、その時は \fbox{\$\_POST = es(\$\_POST)} なんてことをせずに
画面出力の時に \fbox{echo \ es(\$name)} とやればよい \dots\dots 著者の
考えているのはそういうことだろう。

しかし、それで初心者は理解できるだろうか?
初心者は \fbox{\$\_POST = es(\$\_POST)} というコードを丸覚えしてしまうので
はなかろうか? 「\$\_POST は最初に es()関数で HTMLエスケープしてしまったら
簡単で楽だ」と思うのではなかろうか?

また、\fbox{\$\_POST = es(\$\_POST)} と \$\_POST を書き変えてしまっている
のは問題だと思う。せめて、他の変数名で格納してほしかった。
たとえば \fbox{\$P\_DATA = es(\$\_POST)} という具合に。

htmlspecialchars関数の使いどころは、PHPの初心者が一度は悩むところだと
思う。その意味では、ここははっきりと、

\vspace{3mm}
\hspace{6mm}
\underline{\Large 画面出力の時点でhtmlspecialchars関数を使う}
\vspace{3mm}

\noindent
ということを明確にすべきではないかと思う。
そして、画面出力するときにいちいち \\
\hspace{6mm}
\fbox{echo htmlspecialchars(\$str, ENT\_QUOTES, ``UTF-8'');} \\
とやるのは大変だから、たとえば、 \\
\hspace{6mm}
\fbox{echo h(\$str);} \\
というふうに、ユーザー定義関数をつくるのだということを初心者に理解して
もらえればいいと思う。

著者が自分の書くコードで es()関数を使って、\$\_POSTを書き変えて、楽を
するのは勝手だが、それを初心者の方に勧めるのはどうかと思う。

\section{参考}

以下のサイトが参考になる。

\begin{itemize}
 \item \href{https://ja.stackoverflow.com/questions/2408/%E8%84%86%E5%BC%B1%E6%80%A7%E5%AF%BE%E7%AD%96%E3%81%AB%E3%81%8A%E3%81%91%E3%82%8Bhtmlspecialchars%E3%81%AE%E4%BD%BF%E7%94%A8%E7%AE%87%E6%89%80%E3%81%AB%E3%81%A4%E3%81%84%E3%81%A6}
       {脆弱性対策におけるhtmlspecialchars()の使用箇所について}
 \item \href{https://qiita.com/rana_kualu/items/11cd41de5f0364ba2ee8}
       {サニタイズ/入力値検証/エスケープの考え方}
\end{itemize}




\section{じゃあ、どう書けばいいか?}

では、どう書けばいいか。

\subsection{util.php の es()関数 を修正する}

p267 に書かれている util.php の es()関数であるが、\$\_POSTを再帰的に htmlspecialchars()関数を
実行するなんてことは、必要ない。
なぜなら、画面出力の時に htmlspecialchars()関数にかけるのであるから、文字列に対してかけることに
なるからである。配列を配列のまま画面出力することはない。foreach()関数などを使って、文字列を
取り出して画面に出力するからである。

同じ名前の es()関数だと紛らわしいので h()関数とする。以下のようになる。

\begin{lstlisting}[caption=util.php]
 <?php
 function h($data) {
   return htmlspecialchars($data, ENT_QUOTES, "UTF-8");
 }
\end{lstlisting}

Shist\_JIS や EUC、JIS を扱うことはないだろうから、"UTF-8" でいいのではないだろうか?

\subsection{画面に出力するところで使う}

p272 の nameCheck.php は以下のようになる。

\begin{lstlisting}[
 caption=nameChack.php,
 language=php
]
 <!DOCTYPE html>
 <html lang="ja">
 <head>
   <meta charset="UTF-8">
   <title>フォーム入力チェック</title>
   <link href="../../css/style.css" rel="stylesheet">
 </head>
 <body>
   <div>

   <?php
     reuqire_once("../../lib/util.php");
     // 文字エンコードの検証
     if (!cken(%_POST)) {
       $encoding = mb_internal_encoding();
       $err = "Encoding Error! The expected encoding is " . $encoding ;
       // エラーメッセージを出して、以下のコードをすべてキャンセルする
       exit($err);
     }                                         // <== <1>
   ?>

   <?php
     // エラーフラグ
     $isError = false;
     // 名前を取り出す
     if (isset($_POST['name'])) {
       $name = trim($_POST['name']);           // <== <2>
       if ($name === "") {
         // 空白のときエラー
         $isError = true;
       }
     } else {
       // 未設定のときエラー
       $isError = true;
     }
  ?>

 <?php if ($isError): ?>
   ... ( 略 ) ...
 <?php else: ?>
   <!-- エラーがなかったとき -->
   <span>
   こんにちは、<?php echo h($name); ?>さん。     // <== <3>
   </span>
 <?php endif; ?>

 </div>
 </body>
 </html>
\end{lstlisting}

\noindent
\verb!<1>! --- h()関数はここでは使わない。\\
\verb!<2>! --- \$\_POST['name'] を \$name に代入。\\
\verb!<3>! --- 画面に出力するときに h()関数を使う。

\vspace{3mm}
\begin{Large}
この本の中では、いたる所で \fbox{\$\_POST = es(\$\_POST)} としているので、
ほとんど書き直さねばならないということになる。
\end{Large}
\vspace{3mm}

しかし、そんなにたいそうなことでもないので、特に問題にならないと思う。
書き直すことで、初心者の方にはいい勉強になると思う。

\newpage
\theendnotes

\$\_POSTの中身を見るには、以下のコードを実行する。

\begin{lstlisting}[
 caption=showPost.php,
 language=php
]
<?php
require_once('util.php');

// 仮に以下のような $_POST を用意するとして、
$_POST = [
  'name' => '<textarea>悪意</textarea>',
  'text' => '<script>alert("virus")</script>'
];

$_POST = es($_POST);
?>
<!doctype html>
<html lang="ja">
  <head>
    <meta charset="utf-8"/>
    <title>POST</title>
  </head>
  <body>
    <h1>POST</h1>
    <h2>print_rで出力</h2>
    <pre><?php print_r($_POST); ?></pre>
    <h2>echoで出力</h2>
    <?php 
    foreach($_POST as $key => $value) {
      echo  $key, ' ', $value, '<br>', PHP_EOL;
    }
    ?>
  </body>
</html>
\end{lstlisting}

このようにブラウザに出力される。

\begin{tcolorbox}
print\_rで出力 \\

Array \\
( \\
\verb!    [name] => <textarea>悪意</textarea>! \\
\verb!    [text] => <script>alert("virus")</script>! \\
) \\

echoで出力 \\
\verb!name <textarea>悪意</textarea>! \\
\verb!text <script>alert("virus")</script>!
\end{tcolorbox}

しかし、実際は、以下のような文字列になっている。

\fbox{$>$ php showPost.php}

\begin{lstlisting}[
 caption=php showPost.php,
 numbers=none,
 language=php
]
<h2>print_rで出力</h2>
<pre>Array
(
    [name] => &lt;textarea&gt;悪意&lt;/textarea&gt;
    [text] => &lt;script&gt;alert(&quot;virus&quot;)&lt;/script&gt;
)
</pre>
<h2>echoで出力</h2>
name &lt;textarea&gt;悪意&lt;/textarea&gt;<br>
text &lt;script&gt;alert(&quot;virus&quot;)&lt;/script&gt;<br>
\end{lstlisting}


\$\_POST の中味がエスケープされた文字列に置き換っている。



\end{document}

%% 修正時刻: Sat May  2 15:10:04 2020


%% 修正時刻: Thu Feb 17 07:37:42 2022
