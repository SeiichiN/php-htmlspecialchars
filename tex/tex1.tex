\documentclass[uplatex, dvipdfmx]{jsarticle}

\include{begin}


\section{htmlspecialchars関数の働き}

htmlspecialchars関数の働きは、以下のようなものである。

\begin{lstlisting}
 $htmltext = '<div id="wrap"><h1>TEST</h1></div>';
 echo htmlspecialchars($htmltext, ENT_QUOTE, "UTF-8");
\end{lstlisting}

\vspace{3mm}
\begin{tabular}{|c|} \hline
\verb!&lt;div id=&quot;wrap&quot;&gt;&lt;h1&gt;TEST&lt;/h1&gt;&lt;/div&gt;! \\ \hline
\end{tabular}
\vspace{3mm}

これをブラウザで見ると、 \\
\fbox{<div id="wrap"><h1>TEST</h1></div>} \\
 となっている。

 だから、フォームにて、JavaScript や $<$table$>$タグなどの余計な HTMLタグが
 入力されたとしても、それを無力化できる。



 \section{PHP7+MySQL入門ノートでの記述}

 『PHP7+MySQL入門ノート』では、以下のような記述になっている。

\begin{lstlisting}[caption=discount.php (8-5)]
<!DOCTYPE html>
<html lang="ja">
<head>
  <meta charset="utf-8">
  <title>金額の計算</title>
  <link href="../../css/style.css" rel="stylesheet">
</head>
<body>
<div>
<?php
  require_once("../../lib/util.php");
  // 文字エンコードの検証
  if (!cken($_POST)){
    $encoding = mb_internal_encoding();
    $err = "Encoding Error! The expected encoding is " . $encoding ;
    // エラーメッセージを出して、以下のコードをすべてキャンセルする
    exit($err);
  }
  // HTMLエスケープ(XSS対策)
  $_POST = es($_POST);
?>
  ( ...以下、略... )
\end{lstlisting}
% <?php
%   // エラーメッセージを入れる配列
%   $errors = [];
%   //クーポンコード
%   if (isset($_POST['couponCode'])) {
%     $couponCode = $_POST['couponCode'];
%   } else {
%     // 未設定エラー
%     $couponCode = "";
%   }
%   //商品ID
%   if (isset($_POST['goodsID'])) {
%     $goodsID = $_POST['goodsID'];
%   } else {
%     // 未設定エラー
%     $goodsID = "";
%   }
% ?>

% <?php
%   // セールデータを読み込む
%   require_once("saleData.php");
%   // 割引率と単価
%   $discount = getCouponRate($couponCode);
%   $tanka = getPrice($goodsID);
%   // 割引率と単価に値があるかどうかチェックする
%   if (is_null($discount)||is_null($tanka)){
%     // エラーメッセージを出して、以下のコードをすべてキャンセルする
%     $err = '<div class="error">不正な操作がありました。</div>';
%     exit($err);
%   }
% ?>

% <?php
%   // 個数の入力値
%   if(isset($_POST['kosu'])) {
%     $kosu = $_POST['kosu'];
%     // 入力値のチェック
%     if (!ctype_digit($kosu)){
%       // 整数ではないときエラー
%       $errors[] = "個数は整数で入力してください。";
%     }
%   } else {
%     // 未設定エラー
%     $errors[] = "個数が未設定";
%   }
% ?>

% <?php
% if (count($errors)>0){
%   //エラーがあったとき
%   echo '<ol class="error">';
%   foreach ($errors as $value) {
%     echo "<li>", $value , "</li>";
%   }
%   echo "</ol>";
% } else {
%   // エラーがなかったとき(端数は切り捨て)
%   $price = $tanka * $kosu;
%   $discount_price = floor($price * $discount);
%   $off_price = $price - $discount_price;
%   $off_per = (1 - $discount)*100;
%   // 3桁位取り
%   $tanka_fmt = number_format($tanka);
%   $discount_price_fmt = number_format($discount_price);
%   $off_price_fmt = number_format($off_price);
%   // 表示する
%   echo "単価:{$tanka_fmt}円、", "個数:{$kosu}個", "<br>";
%   echo "金額:{$discount_price_fmt}円", "<br>";
%   echo "(割引:-{$off_price_fmt}円、{$off_per}% OFF)", "<br>";
% }
% ?>

% <!-- 戻りボタンのフォーム -->
%   <form method="POST" action="discountForm.php">
%     <!-- 隠しフィールドに個数を設定してPOSTする -->
%     <input type="hidden" name="kosu" value="<?php echo $kosu; ?>">
%     <ul>
%       <li><input type="submit" value="戻る"></li>
%     </ul>
%   </form>

% </div>
% </body>
% </html> 
 
この著者のやり方では、\$\_POST データが送られてきたら、まず、「文字エンコードの検証」を
おこない(13行目)、次に「HTMLエスケープ」をおこなっている(20行目)。

「文字エンコードの検証」は必要だと思うが、「HTMLエスケープ」は、\$\_POSTデータを取得直後にするべきだろうか。

特に問題だと思われるのは、20行目である。

\fbox{\$\_POST = es(\$\_POST)}

\$\_POST の中味を書き変えている。

この es関数がどのようなものかというと、

\begin{lstlisting}[caption=util.php]
<?php
// XSS対策のためのHTMLエスケープ
function es($data, $charset='UTF-8'){
  // $dataが配列のとき
  if (is_array($data)){
    // 再帰呼び出し
    return array_map(__METHOD__, $data);
  } else {
    // HTMLエスケープを行う
    return htmlspecialchars($data, ENT_QUOTES, $charset);
  }
}
        ( ...以下、略... )
\end{lstlisting}

% // 配列の文字エンコードのチェックを行う
% function cken(array $data){
%   $result = true;
%   foreach ($data as $key => $value) {
%     if (is_array($value)){
%       // 含まれている値が配列のとき文字列に連結する
%       $value = implode("", $value);
%     }
%     if (!mb_check_encoding($value)){
%       // 文字エンコードが一致しないとき
%       $result = false;
%       // foreachでの走査をブレイクする
%       break;
%     }
%   }
%   return $result;
% }
% // ?> 


\$\_POST の中を再帰的に htmlspecialchars関数を実行している。

たとえば、以下のような \$\_POST データが送られてきたとする。

\begin{lstlisting}
 $_POST = [
  'name' => '<textarea>悪意</textarea>',
  'text' => '<script>alert("virus")</script>'
];
\end{lstlisting}

これを以下のコードで実行する。

\begin{lstlisting}
<?php
require_once('util.php');

$_POST = [
  'name' => '<textarea>悪意</textarea>',
  'text' => '<script>alert("virus")</script>'
];

$_POST = es($_POST);
?>
<!doctype html>
<html lang="ja">
  <head>
    <meta charset="utf-8"/>
    <title></title>
  </head>
  <body>
    <h1></h1>
    <h2>print_rで出力</h2>
    <pre><?php print_r($_POST); ?></pre>
    <h2>echoで出力</h2>
    <?php 
    foreach($_POST as $key => $value) {
      echo  $key, ' ', $value, '<br>', PHP_EOL;
    }
    ?>
    <script>
     'use strict';

    </script>
  </body>
</html>
\end{lstlisting}

このようにブラウザに出力される。

\begin{tcolorbox}
print\_rで出力 \\

Array \\
( \\
\verb!    [name] => <textarea>悪意</textarea>! \\
\verb!    [text] => <script>alert("virus")</script>! \\
) \\

echoで出力 \\
\verb!name <textarea>悪意</textarea>! \\
\verb!text <script>alert("virus")</script>!
\end{tcolorbox}

しかし、実際は、以下のような文字列になっている。

\begin{verbatim}
<h2>print_rで出力</h2>
<pre>Array
(
    [name] => &lt;textarea&gt;悪意&lt;/textarea&gt;
    [text] => &lt;script&gt;alert(&quot;virus&quot;)&lt;/script&gt;
)
</pre>
<h2>echoで出力</h2>
name &lt;textarea&gt;悪意&lt;/textarea&gt;<br>
text &lt;script&gt;alert(&quot;virus&quot;)&lt;/script&gt;<br>
\end{verbatim}

つまり、\$\_POST の中味がエスケープされた文字列に置き換っているのである。

ここでは、\$\_POST の中味をすぐに画面に出力しているからいいが、
これを MySQL などに保存するとなると、大事になる。

\section{このやり方の良くないところ}

ここでの著者のやり方は、\$\_POST でデータが送られてきたら、とりあえず、
htmlspecialchars関数を使って \$\_POST を安全なものにしてしまおうという
やり方である。

初心者の人にこのやり方を教えれば、この通りにすぐに htmlspecialchars関数を使って
同じようにやってしまうだろう。

しかし、本来は、htmlspecialchars関数は、画面に出力するタイミングで行うもので
なければならない。
この著者のやり方では、間違ったタイミングを教えてしまうことになる。

更に良くないのは、\$\_POST を書き変えてしまう点である。
元のデータは大事にしなければならない。これは避けるべきだと思う。



\include{end}

%% 修正時刻: Mon Feb 14 07:27:15 2022
