\documentclass[uplatex, dvipdfmx]{jsarticle}


\usepackage{tcolorbox}
\usepackage{color}
\usepackage{listings, plistings}

%% ノート/latexメモ
%% http://pepper.is.sci.toho-u.ac.jp/pepper/index.php?%A5%CE%A1%BC%A5%C8%2Flatex%A5%E1%A5%E2

% Java
\lstset{% 
  frame=single,
  backgroundcolor={\color[gray]{.9}},
  stringstyle={\ttfamily \color[rgb]{0,0,1}},
  commentstyle={\itshape \color[cmyk]{1,0,1,0}},
  identifierstyle={\ttfamily}, 
  keywordstyle={\ttfamily \color[cmyk]{0,1,0,0}},
  basicstyle={\ttfamily},
  breaklines=true,
  xleftmargin=0zw,
  xrightmargin=0zw,
  framerule=.2pt,
  columns=[l]{fullflexible},
  numbers=left,
  stepnumber=1,
  numberstyle={\scriptsize},
  numbersep=1em,
  language={Java},
  lineskip=-0.5zw,
  morecomment={[s][{\color[cmyk]{1,0,0,0}}]{/**}{*/}},
  keepspaces=true,         % 空白の連続をそのままで
  showstringspaces=false,  % 空白字をOFF
}
%\usepackage[dvipdfmx]{graphicx}
\usepackage{url}
\usepackage[dvipdfmx]{hyperref}
\usepackage{amsmath, amssymb}
\usepackage{itembkbx}
\usepackage{eclbkbox}	% required for `\breakbox' (yatex added)
\usepackage{enumerate}
\usepackage[default]{cantarell}
\usepackage[T1]{fontenc}
\usepackage{endnotes}
\fboxrule=0.5pt
\parindent=1em

\makeatletter
\def\verbatim@font{\normalfont
\let\do\do@noligs
\verbatim@nolig@list}
\makeatother

\begin{document}

%\anaumeと入力すると穴埋め解答欄が作れるようにしてる。\anaumesmallで小さめの穴埋めになる。
\newcounter{mycounter} % カウンターを作る
\setcounter{mycounter}{0} % カウンターを初期化
\newcommand{\anaume}[1][]{\refstepcounter{mycounter}{#1}{\boxed{\phantom{aa}\textnormal{\themycounter}\phantom{aa}}}} %穴埋め問題の空欄作ってる。
\newcommand{\anaumesmall}[1][]{\refstepcounter{mycounter}{#1}{\boxed{\tiny{\phantom{a}\themycounter \phantom{a}}}}}%小さい版作ってる。色々改造できる。

%% 修正時刻: Wed Feb 16 06:54:13 2022



\section{htmlspecialchars関数の働き}

htmlspecialchars関数の働きは、以下のようなものである。

\begin{lstlisting}
 $htmltext = '<div id="wrap"><h1>TEST</h1></div>';
 echo htmlspecialchars($htmltext, ENT_QUOTE, "UTF-8");
\end{lstlisting}

\begin{tabular}{|c|}
\verb!&lt;div id=&quot;wrap&quot;&gt;&lt;h1&gt;TEST&lt;/h1&gt;&lt;/div&gt;! \\
\end{tabular}

これをブラウザで見ると、 \\
\fbox{<div id="wrap"><h1>TEST</h1></div>} \\
 となっている。

 だから、フォームにて、JavaScript や $<$table$>$タグなどの余計な HTMLタグが
 入力されたとしても、それを無力化できるのである。


 \section{PHP7+MySQL入門ノートでの記述}

 『PHP7+MySQL入門ノート』では、以下のような記述になっている。

\begin{lstlisting}[caption=discount.php (8-5)]
<!DOCTYPE html>
<html lang="ja">
<head>
  <meta charset="utf-8">
  <title>金額の計算</title>
  <link href="../../css/style.css" rel="stylesheet">
</head>
<body>
<div>
<?php
  require_once("../../lib/util.php");
  // 文字エンコードの検証
  if (!cken($_POST)){
    $encoding = mb_internal_encoding();
    $err = "Encoding Error! The expected encoding is " . $encoding ;
    // エラーメッセージを出して、以下のコードをすべてキャンセルする
    exit($err);
  }
  // HTMLエスケープ(XSS対策)
  $_POST = es($_POST);
?>

<?php
  // エラーメッセージを入れる配列
  $errors = [];
  //クーポンコード
  if (isset($_POST['couponCode'])) {
    $couponCode = $_POST['couponCode'];
  } else {
    // 未設定エラー
    $couponCode = "";
  }
  //商品ID
  if (isset($_POST['goodsID'])) {
    $goodsID = $_POST['goodsID'];
  } else {
    // 未設定エラー
    $goodsID = "";
  }
?>

<?php
  // セールデータを読み込む
  require_once("saleData.php");
  // 割引率と単価
  $discount = getCouponRate($couponCode);
  $tanka = getPrice($goodsID);
  // 割引率と単価に値があるかどうかチェックする
  if (is_null($discount)||is_null($tanka)){
    // エラーメッセージを出して、以下のコードをすべてキャンセルする
    $err = '<div class="error">不正な操作がありました。</div>';
    exit($err);
  }
?>

<?php
  // 個数の入力値
  if(isset($_POST['kosu'])) {
    $kosu = $_POST['kosu'];
    // 入力値のチェック
    if (!ctype_digit($kosu)){
      // 整数ではないときエラー
      $errors[] = "個数は整数で入力してください。";
    }
  } else {
    // 未設定エラー
    $errors[] = "個数が未設定";
  }
?>

<?php
if (count($errors)>0){
  //エラーがあったとき
  echo '<ol class="error">';
  foreach ($errors as $value) {
    echo "<li>", $value , "</li>";
  }
  echo "</ol>";
} else {
  // エラーがなかったとき(端数は切り捨て)
  $price = $tanka * $kosu;
  $discount_price = floor($price * $discount);
  $off_price = $price - $discount_price;
  $off_per = (1 - $discount)*100;
  // 3桁位取り
  $tanka_fmt = number_format($tanka);
  $discount_price_fmt = number_format($discount_price);
  $off_price_fmt = number_format($off_price);
  // 表示する
  echo "単価:{$tanka_fmt}円、", "個数:{$kosu}個", "<br>";
  echo "金額:{$discount_price_fmt}円", "<br>";
  echo "(割引:-{$off_price_fmt}円、{$off_per}% OFF)", "<br>";
}
?>

<!-- 戻りボタンのフォーム -->
  <form method="POST" action="discountForm.php">
    <!-- 隠しフィールドに個数を設定してPOSTする -->
    <input type="hidden" name="kosu" value="<?php echo $kosu; ?>">
    <ul>
      <li><input type="submit" value="戻る"></li>
    </ul>
  </form>

</div>
</body>
</html> 
\end{lstlisting}
 
この著者のやり方では、\$\_POST データが送られてきたら、まず、「文字エンコードの検証」を
おこない(13行目)、次に「HTMLエスケープ」をおこなっている(20行目)。

特に問題だと思われるのは、20行目である。

\fbox{\$\_POST = es(\$\_POST)}

\$\_POST の中味を書き変えてしまっているのである。

この es関数がどのようなものかというと、

\begin{lstlisting}[caption=util.php]
<?php
// XSS対策のためのHTMLエスケープ
function es($data, $charset='UTF-8'){
  // $dataが配列のとき
  if (is_array($data)){
    // 再帰呼び出し
    return array_map(__METHOD__, $data);
  } else {
    // HTMLエスケープを行う
    return htmlspecialchars($data, ENT_QUOTES, $charset);
  }
}

// 配列の文字エンコードのチェックを行う
function cken(array $data){
  $result = true;
  foreach ($data as $key => $value) {
    if (is_array($value)){
      // 含まれている値が配列のとき文字列に連結する
      $value = implode("", $value);
    }
    if (!mb_check_encoding($value)){
      // 文字エンコードが一致しないとき
      $result = false;
      // foreachでの走査をブレイクする
      break;
    }
  }
  return $result;
}
// ?> 
\end{lstlisting}

\$\_POST の中を再帰的に htmlspecialchars関数を実行している。

たとえば、以下のような \$\_POST データが送られてきたとする。

\begin{lstlisting}
 $_POST = [
  'name' => '<textarea>悪意</textarea>',
  'text' => '<script>alert("virus")</script>'
];
\end{lstlisting}

これを以下のコードで実行する。

\begin{lstlisting}
<?php
require_once('util.php');

$_POST = [
  'name' => '<textarea>悪意</textarea>',
  'text' => '<script>alert("virus")</script>'
];

$_POST = es($_POST);
?>
<!doctype html>
<html lang="ja">
  <head>
    <meta charset="utf-8"/>
    <title></title>
  </head>
  <body>
    <h1></h1>
    <h2>print_rで出力</h2>
    <pre><?php print_r($_POST); ?></pre>
    <h2>echoで出力</h2>
    <?php 
    foreach($_POST as $key => $value) {
      echo  $key, ' ', $value, '<br>', PHP_EOL;
    }
    ?>
    <script>
     'use strict';

    </script>
  </body>
</html>
\end{lstlisting}

このようにブラウザに出力される。

\begin{tcolorbox}
print\_rで出力 \\

Array \\
( \\
\verb!    [name] => <textarea>悪意</textarea>! \\
\verb!    [text] => <script>alert("virus")</script>! \\
) \\

echoで出力 \\
\verb!name <textarea>悪意</textarea>! \\
\verb!text <script>alert("virus")</script>!
\end{tcolorbox}

しかし、実際は、以下のような文字列になっている。

\begin{verbatim}
<h2>print_rで出力</h2>
<pre>Array
(
    [name] => &lt;textarea&gt;悪意&lt;/textarea&gt;
    [text] => &lt;script&gt;alert(&quot;virus&quot;)&lt;/script&gt;
)
</pre>
<h2>echoで出力</h2>
name &lt;textarea&gt;悪意&lt;/textarea&gt;<br>
text &lt;script&gt;alert(&quot;virus&quot;)&lt;/script&gt;<br>
\end{verbatim}

つまり、\$\_POST の中味がエスケープされた文字列に置き換っているのである。

ここでは、\$\_POST の中味をすぐに画面に出力しているからいいが、
これを MySQL などに保存するとなると、大事になる。

\section{このやり方の危険なところ}

ここでの著者のやり方は、\$\_POST でデータが送られてきたら、とりあえず、
htmlspecialchars関数を使って \$\_POST を安全なものにしてしまおうという
やり方である。

初心者の人にこのやり方を教えれば、この通りにすぐに htmlspecialchars関数を使って
同じようにやってしまうだろう。

しかし、本来は、htmlspecialchars関数は、画面に出力するタイミングで行うもので
なければならない。
この著者のやり方では、間違ったタイミングを教えてしまうことになる。

更に危険なのは、\$\_POST を書き変えてしまう点である。
元のデータは大事にしなければならない。これは避けるべきである。



\end{document}

%% 修正時刻: Sat May  2 15:10:04 2020


%% 修正時刻: Sat Feb 12 09:57:22 2022
